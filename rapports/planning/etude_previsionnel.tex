\documentclass[12pt,a4paper]{article}
\usepackage[utf8]{inputenc}
\usepackage[french]{babel}
\usepackage[T1]{fontenc}
\usepackage{amsmath}
\usepackage{amsfonts}
\usepackage{amssymb}
\usepackage{graphicx}
\author{KONDI Abdoul malik \\ NGANDEU NDJEUKAM Alhasan}
\title{Étude prévisionnel du projet de gestion d'une librairie}
\begin{document}
\maketitle
\tableofcontents
\newpage

\section{Planning}

image

\subsection{Résumé récapitulatif du planning:}
\begin{flushleft}

\begin{tabular}{|c|c|c|}
\hline 
Début & Fin & Tâche à accomplir \\ 
\hline 
\textbf{01/08/2022} & \textbf{13/08/2022} & Réalisation du modèle \\ 
\hline 
\textbf{15/08/2022} & \textbf{17/08/2022} & Initialisation du projet et interaction avec la base de données \\ 
\hline
\textbf{18/08/2022} & \textbf{22/08/2022} & Implémentation de certains cas d'utilisation \\ 
\hline 
\textbf{24/08/2022} & \textbf{29/08/2022} & Système d'authentification et interaction avec l'application web\\ 
\hline
\textbf{30/08/2022} & \textbf{02/09/2022} & Configuration de la machine distante et déploiement\\ 
\hline 
\textbf{05/09/2022} & \textbf{07/09/2022} & Designer l'application \\ 
\hline 
\textbf{08/09/2022} & \textbf{12/09/2022} & Écriture du manuel d'utilisation \\ 
\hline
\textbf{12/09/2022} & \textbf{14/09/2022} & Apprendre l'utilisation de l'application aux utilisateurs \\ 
\hline  
\textbf{14/09/2022} & \textbf{16/09/2022} & Réalisation de l'application mobile(android \& ios) \\ 
\hline 
 

\end{tabular} 
\end{flushleft}















\end{document}




