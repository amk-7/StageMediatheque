\documentclass[12pt,a4paper]{article}
\usepackage[utf8]{inputenc}
\usepackage[french]{babel}
\usepackage[T1]{fontenc}
\usepackage{amsmath}
\usepackage{amsfonts}
\usepackage{amssymb}
\usepackage{graphicx}
\usepackage[left=2cm,right=2cm,top=2cm,bottom=2cm]{geometry}
\author{KONDI Abdoul malik \\ NGANDEU NDJEUKAM Alhasan}
\title{Manuelle d'utilisation de l'application de gestion de la médiathèque}
\begin{document}
\maketitle
\tableofcontents
\newpage

\section{Préambule}
Ce manuel d'utilisation ci explique comment utilisé le site web de la médiathèque afin
de permettre aux personnels de la médiathèque de ce familiariser avec ce dernier.

\newpage
\section{Se connecter à l'application}
Pour se connecter à l'application, cliquer sur le bouton se connecter.\\
Une nouvelle page vient d'apparaître. Renseigner votre e mail et votre mot de passe 
puis cliquer sur le bouton \textbf{SE CONNECTER}. C'est tous maintenant vous êtes 
connecté.

\newpage
\section{Accès au menu déroulant}
Accéder au menu déroulant comme suit : 
\subsection{Étape 1} 
\begin{itemize}
\item[•] Cliquer sur les trois barres en haut à gauche 
\end{itemize}
\includegraphics[scale=0.5]{images/SelectOuvragePhysique.png}

\subsection{Etape 2}
\begin{itemize}
\item[•] Après avoir cliquer sur les trois barres vous obtiendrais le résultat ci-dessous
\end{itemize}
\includegraphics[scale=0.7]{images/TableauDeBord.png}.\\
Après cette étape vous pouvez sélectionner les options et les utiliser.\\
Débutons avec la \textbf{Gestion des ouvrages}.

\newpage
\section{Gestion des ouvrages}
Le menu gestion des ouvrages nous permet de gérer tous les types d'ouvrages à savoir les ouvrages physiques comme électroniques. De même la gestion des ouvrages nous permet de gérer les emprunts et restitutions.
Allons dans le menu déroulant et choisissons le menu ouvrages physiques. \\
\includegraphics[scale=0.7]{images/TableauDeBord2.png}

\subsection{Accès à l'ouvrage physique}
Menu Ouvrage physique.\\

\includegraphics[scale=0.35]{images/SelectOuvragesPhysique.png}. \\

Dans le menu ouvrage physique nous pouvons ajouter un ouvrage, l'éditer, le consulter ou le supprimer.

\begin{itemize}
\item[•]Etape 1: Ajoutons un ouvrage physique.
\end{itemize}

Pour ce faire nous allons cliquer sur le boutons \textbf{ajouter} 

\includegraphics[scale=0.5]{images/SelectOuvragesPhysique2.png}. \\

\begin{itemize}
\item[•]Etape 2 : Remplissez les champs suivant avec des données conformes.
\end{itemize}

Comme champs à remplir nous avons :\\
\textbf{La section Ouvrage} qui contient le titre de l'ouvrage, l'image de l'ouvrage, le niveau de l'ouvrage, son type, la langue, année d'apparition et le lieu d’édition.\\
\includegraphics[scale=0.5]{images/AjoutOuvrage.png}.\\
\newpage

\textbf{La section Auteur} qui contient le nom et prénom de ou des auteurs.\\
\textbf{La section Particularité} qui contint la catégorie de l'ouvrage ainsi que l'ISBN.\\
\textbf{La section Mot clé} qui contient les mots clé de l'ouvrage. \\

\includegraphics[scale=0.5]{images/AuteurParticulariteMotCle.png}.\\

\textbf{La section Résumé de l'ouvrage} qui contient le résumé de l'ouvrage.\\
\textbf{La section stock} qui contient le nombre d'exemplaire à la disposition de la bibliothèque, le rayon ainsi que l'étagère où se situe l'ouvrage.\\

\includegraphics[scale=0.5]{images/ResumeStock.png}.\\

\textbf{Après avoir entré toute les informations vous cliquez sur le boutons enregistrer afin de valider les informations entrée}.\\

\newpage
\textbf{Dans la gestion des ouvrages vous pouvez consulter les ouvrages en cliquant sur le bouton consulté}. \\

\includegraphics[scale=0.5]{images/ConsultationOuvragePhysique.png}.

Lorsque vous cliquez sur consulter vous avez la possibilité de voir les détails de l'ouvrage et vous pouvez également réserver cet ouvrage s'il n'es pas encore pris.\\

\includegraphics[scale=0.5]{images/ResulConsulter.png}.\\

\newpage

\textbf{Dans le cas où vous avez mal enregistré un ouvrage vous pouvez le modifier en cliquant sur le bouton éditer}.\\

\includegraphics[scale=0.5]{images/ResulEdit.png}.\\

Ainsi lorsque vous cliquez sur éditer vous aller vous retrouver à la page d'édition qui ressemble beaucoup à celle de l'ajout d'un ouvrage mais la seule différence est que les données seront déjà rempli et vous n'avez juste qu'à les modifier. \\

\textbf{La dernière possibilité est la suppression d'un ouvrage}
Pour ce faire vous n'avez juste qu'à cliquer sur le bouton supprimer et l'ouvrage sera automatiquement supprimé.\\

\includegraphics[scale=0.5]{images/ResulSuppression.png}.\\

\newpage
\subsection{Accès à l'Ouvrage Électronique}
Menu Ouvrage Électronique.\\
Dans un premier temps accéder au menu gestion des ouvrages puis ensuite sélectionner ouvrage électronique.\\
\includegraphics[scale=0.4]{images/OuVELectro.png}.\\

Après avoir cliquer nous serons sur le menu ouvrage électronique.\\

\includegraphics[scale=0.4]{images/TashBoardElectro.png}.\\

Dans le menu ouvrage électronique nous pouvons ajouter un ouvrage, l'éditer, le consulter ou le supprimer.

\begin{itemize}
\item[•]Etape 1: Ajoutons un ouvrage électronique.
\end{itemize}

Pour ce faire nous allons cliquer sur le boutons \textbf{ajouter} 

\includegraphics[scale=0.5]{images/BoutonsAjout.png}. \\

\begin{itemize}
\item[•]Etape 2 : Remplissez les champs suivant avec des données conformes.
\end{itemize}

Comme champs à remplir nous avons :\\
\textbf{La section Ouvrage} qui contient le titre de l'ouvrage, l'image de l'ouvrage, le niveau de l'ouvrage, son type, la langue, année d'apparition et le lieu d’édition.\\
\includegraphics[scale=0.5]{images/Ajoutuvragelectro.png}.\\
\newpage

\textbf{La section Auteur} qui contient le nom et prénom de ou des auteurs.\\
\textbf{La section Particularité} qui contint la catégorie de l'ouvrage ainsi que l'ISBN.\\
\textbf{La section Mot clé} qui contient les mots clé de l'ouvrage. \\

\includegraphics[scale=0.5]{images/APMelectro.png}.\\

\textbf{La section Résumé de l'ouvrage} qui contient le résumé de l'ouvrage.\\
\textbf{La section fichier pdf} qui nous permettra de charger nos fichier pdf à partir de notre machine et les insérer dans l'application.\\

\includegraphics[scale=0.5]{images/ResumeFichier.png}.\\

\textbf{Après avoir entré toute les informations vous cliquez sur le boutons enregistrer afin de valider les informations entrée}.\\

\textbf{Pour ce qui est des options consulter, modifier et supprimer c'est pareil que celui de l'ouvrage physique}.\\

\newpage
\section{Accès à l'Emprunt}
Dans le barre de navigation cliquez sur le menu emprunt et vous serez dirigé vers le menu emprunt.\\

\includegraphics[scale=0.35]{images/Emprunt.png}.\\

Dans le menu Emprunt nous pouvons ajouter un emprunt, l'éditer, le consulter, le restituer ou le supprimer.

\begin{itemize}
\item[•]Etape 1: Ajoutons un Emprunt.
\end{itemize}

Pour ce faire nous allons cliquer sur le boutons \textbf{ajouter} 

\begin{itemize}
\item[•]Etape 2 : Remplissez les champs suivant avec des données conformes.
\end{itemize}

Comme champs à remplir nous avons :\\
\textbf{La section Abonne} qui contient le nom et prénom de l'abonne.\\

\includegraphics[scale=0.5]{images/Abonne.png}

\textbf{La section Ouvrage} qui contient la côte, le titre ainsi que l'état de l'ouvrage.\\

\includegraphics[scale=0.5]{images/EOuvrage.png}.\\

\textbf{La section Durée} qui contient la période à laquelle le livre a été prêté.\\
Pour cette partie lorsque vous finissez de remplir ce champs veuillez à cliquer sur le bouton ajouter vous verrez que l'emprunt à été ajouter et vous pouvez également ajouter un autre emprunt si l'abonné veux prendre par exemple deux ouvrages.\\

\includegraphics[scale=0.5]{images/EDureeAjout.png}.\\
Après l'avoir fait vous n'avez plus qu'à cliquez sur le bouton emprunter pour emprunter l'ouvrage.\\

\includegraphics[scale=0.5]{images/EDureeEmprunt.png}.\\

\textbf{Lorsque vous décidez de consulter un emprunt en cours vous n'avez qu'à cliquez sur le bouton consulter}.\\

\includegraphics[scale=0.5]{images/ConsultEmprunt.png}.\\

Et comme résultat vous obtiendrai le nom et prénom de l'abonne ainsi que celui du personnel ayant effectué l'emprunt, le nombre d'ouvrage emprunté, la date d'emprunt et de retour.\\

\includegraphics[scale=0.5]{images/ConsultationEmprunt.png}.\\

\newpage
\textbf{Dans le cas de la modification d'un Emprunt}.\\
Pour la modification de l'emprunt le plus important c'est le fait de prolonger l'emprunt de l'abonné.\\

\includegraphics[scale=0.5]{images/EmpruntEdit.png}.\\

\textbf{Dans le cas de la restitution}.\\
Pour ce qui est de la restitution vous devez cliquer sur le bouton restitution.

\includegraphics[scale=0.5]{images/RestitutionSectionner.png}.
\newpage
Après avoir sélectionnez le bouton restituer vous serez rediriger vers la page de restitution.\\

\includegraphics[scale=0.5]{images/Restitution.png}.\\

Pour effectuer la restitution vous devez cocher la case restitution afin de sélectionner l'état dans lequel le livre est restituer. Procédez comme suit :

\begin{itemize}
\item[•] Etape 1 : Cocher la case blanche au niveau de la colonne restituer.\\
\end{itemize}

\includegraphics[scale=0.5]{images/Cocher.png}.\\

\newpage
Après avoir coché vous serez redirigé sur la page suivante : \\

\includegraphics[scale=0.5]{images/SelectEtat.png}.\\


\begin{itemize}
\item[•] Etape 2 : Sélectionnez l'état du livre puis valider en cliquant sur modifier.
\end{itemize}

\includegraphics[scale=0.5]{images/Restitutionbouton.png}.\\

\newpage
\begin{itemize}
\item[•] Etape 3 : Cliquer sur restituer pour restituer l'ouvrage.\\
\end{itemize}

\includegraphics[scale=0.5]{images/ListeDesRestitutions.png}

\section{Accès à la restitution}
Dans le barre de navigation cliquez sur le menu restitution et vous serez dirigé vers le menu restitution.\\

\includegraphics[scale=0.5]{images/ListeDesRestitutions.png}.

Ici comme option vous pouvez consulter toute les informations relative à la restitution.\\
\textbf{Accédons à l'option consulter de la restitution}.\\

Comme information nous avons le nom et prénom de l'abonné ainsi que tu personnel ainsi que tous les détails relative à l'emprunt de l'ouvrage notamment la côte, le titre, l'état de sorti et de retour de l'ouvrage et si l'ouvrage est totalement restituer ou partiellement.

\includegraphics[scale=0.5]{images/showRestitution.png}

















\newpage
\section{Gestion des utilisateurs}
Le menu gestion des utilisateurs comme vous pouvez facilement le deviné est un menu qui
nous permet de gérer les abonnés, leurs abonnements et le personnel (tous ce qui se
connecterons à l'application). Pour accéder à ce menu, devez cliquer sur le bouton sur le
quelle trois (3) traits on été marqué il se trouve à gauche de l'écran dans une barre
vertical. \\

::::::::::::image requise::::::::::::\\

Une fois que vous cliquer sur ce bouton, vous remarquerez qu'un quart (1/4) de page 
sortira de la gauche de l'écran. Rendez vous maintenant au menu \textbf{Gestion des
utilisateurs} écrit en jaune. Vous y verrez trois sous menus :
\begin{itemize}
\item[•] Abonnés
\item[•] Abonnements
\item[•] Personnels
\end{itemize}

\subsubsection{Sous menu \textbf{Abonnés}}
Cliquer sur \textbf{Abonnés} vous verrez cette page s'afficher.\\
::::::::::::image requise::::::::::::\\
Elle contient un bouton \textbf{AJOUTER} et la liste des abonnés afficher dans un tableau.
Pour chaque abonnés vous aurez trois boutons : \textbf{CONSULTER}, \textbf{ÉDITER}, 
\textbf{SUPPRIMER}.\\
::::::::::::image requise::::::::::::\\
\begin{itemize}
\item[•] Bouton \textbf{AJOUTER} : \\
Il permet l'enregistrement d'un nouvelle abonné. Cliquer dessus. C'est fait ? Vous verrez
une nouvelle page apparaître. Elle vous demande de renseigner les informations de l'abonné
puis de cliquer sur le bouton \textbf{ENREGISTRER} pour enregistrer l'abonné.
vous pouvez revenir à la page précédente comme ceci :\\
::::::::::::image requise::::::::::::\\
ou enregistrer un abonné comme ceci  :\\
::::::::::::image requise::::::::::::\\
\textbf{NB:} Tous les champs sont obligatoire pour enregistrer un abonné.
\item[•] Bouton \textbf{EDITER} : \\
Il permet de modifier une ou toute les informations de l'abonné ciblé. Cliquer dessus. Vous verrez s'afficher quelque chose de similaire.
::::::::::::image requise::::::::::::\\
Si nous voulons par exemple changer le numéro de la personne à prévenir pour quelconque 
raison, nous pouvons le ressaisir comme ceci et cliquer sur le bouton modifier.
::::::::::::image requise::::::::::::\\
Voici le résultat.\\
::::::::::::image requise::::::::::::\\
Vous remarquerez que le contact a été modifier.
\item[•] Bouton \textbf{CONSULTER} : \\
Il permet d'accéder à toute les informations de l'abonné ciblé. Cliquer dessus. Vous verrez
s'afficher quelque chose de similaire.
::::::::::::image requise::::::::::::\\
\item[•] Bouton \textbf{SUPPRIMER} : \\
Il permet de supprimer l'abonné ciblé. Cliquer dessus. 
::::::::::::image requise::::::::::::\\
Vous constatez que l'abonné à belle et bien été supprimé.
\end{itemize}
\subsubsection{Sous menu \textbf{Abonnement}}
Ce sous menu permet de garder les traces des payements des abonnés pour leur
abonnement. Après avoir finis d'enregistrer un abonne, vous pouvez enregistrer le 
payement qu'il a effectuer pour l'abonnement (200 FCFA ou 500 FCFA). Pour le faire :
cliquer sur le menu \textbf{Abonnement}. Sélectionner le nom de l'abonné et le tarifs
d'abonnement puis cliquer sur le bouton enregistrer.\\
::::::::::::image requise::::::::::::\\
Cela enregistrera l'opération et vous pouvez voir le résultat. Une fois l'abonnement
enregistrer vous ne pouvez que le consulter, vous ne pouvez pas le supprimer ni même
l'éditer.

\section{Import excel}
Pour enregistrer plusieurs livre papier via une feuille excel, rendez vous sur le
tableau de bord puis sur le menu \textbf{Import excel} choisissez le sous menu 
\textbf{Livre papier}. Une page s'ouvre. cliquer sur le bouton choisir un fichier,
choisissez votre fichier excel puis appuyez sur le bouton importer. 

\textbf{NB:} Le fichier excel ne doit contenir qu'une seul feuille de calcul répondant
aux format çi dessous.\\
::::::::::Insérer format::::::::::

\end{document}



