\documentclass[12pt,a4paper]{article}
\usepackage[utf8]{inputenc}
\usepackage[french]{babel}
\usepackage[T1]{fontenc}
\usepackage{amsmath}
\usepackage{amsfonts}
\usepackage{amssymb}
\usepackage{graphicx}
\usepackage[left=2cm,right=2cm,top=2cm,bottom=2cm]{geometry}
\author{KONDI Abdoul malik \\ NGANDEU NDJEUKAM Alhasan}
\title{Rapport du stage de fin L2}
\begin{document}
\maketitle
\tableofcontents
\newpage
\section{Préambule}
\section{Sommaire}
\section{Introduction}
\section{Organisme d'accueil}
%présenter l'organisation d'accueil 
%organimagramme complet identifiant le service dans le quelle le stagier à travailé.
\newpage
\section{Analyse et modélisation du problème}
L'objectif de notre stage à la médiathèque était de mettre en place une bibliothèque accessible en ligne. Pour pouvoir y arriver, nous avons du principalement dans les deux première semaines du stage travailler avec le personnel de la bibliothèque pour pouvoir construire un modèle solide. A l'issue 
de ce travaille nous avons réaliser un diagramme de cas d'utilisation et un 
diagramme de classe. Ces diagrammes (surtout le diagramme de classe) ont évolué
au cours du stage. Dans cette parie nous entrerons plus en détails dans chacune 
d'elle.
\subsection{Diagramme de cas d'utilisation}
Le diagramme de cas d'utilisation comme vous le savez déjà décrit toute les actions
possible pour chaque personne (acteur) ayant accès à l'application (au système). Dans 
notre cas pour réaliser ce diagramme nous nous sommes basé essentiellement sur les besoins décrit dans le cahier de charge à savoir : 
\begin{itemize} 
\item[•] Enregistrer de nouvels abonnés et de les classer dans une catégorie (élève, étudiant, professionnel...) ;
\item[•] Permettre au gérant de la médiathèque de gérer les prêts des ouvrages physiques pour les abonnés ;
\item[•] Possibilité pour les utilisateurs de consulter en ligne plus de 2000 livres ;
\item[•] Pouvoir déclencher la recherche d’un livre à l’aide du titre, des mots-clés, de l’auteur, de l’année
de parution du livre, de son type, du domaine auquel il s’applique etc... ;
\item[•] Donner la possibilité aux utilisateurs de la bibliothèque numérique de télécharger des livres au
format PDF ;
\item[•] Pouvoir ajouter de nouveaux livres en saisissant l’ISBN ;
\end{itemize}
Plus tard pendant la phase d'analyse ont été ajouté les besoins suivant.\\
\begin{itemize}
\item[•] La gestion des réservations en ligne de livre physique.
\item[•] L'implémentation d'un serveur de messagerie pour alerter les abonnés afin
que ces dernier puissent ramener les ouvrages emprunté à temps.
\end{itemize}
\includegraphics[scale=0.5]{images/UC.png}
\subsection{Diagramme de classe}
Le diagramme de classe quant à lui nous permet de représenter les données que nous 
avons afin de mieux les stocker dans une base de données et optimiser les multiples
façons de les utiliser. Ici aussi nous nous sommes réfère au cahier de charge et 
aux règles de gestion métier aux quelle nous avons apporté quelque ajout et changement.
Les voici :
\begin{itemize}
\item[•] Un ouvrage à un titre, un niveau, un type, une image, une langue, un résumer et
plusieurs mot clé.
\item[•] Un ouvrage peut être écrit par au moins un auteur et inversement un auteur peut
écrire plusieurs livres.
\item[•] Nous avons des ouvrages physique que nous pouvons approvisionner (qui sont
classé et rangé dans la bibliothèque) et des ouvrages électronique (qu'on peut 
télécharger) qui sont chacun pris individuellement des ouvrages.
\item[•] Parmi les ouvrages physique on compte les livres papier et documents
audiovisuelle et parmi les ouvrages électronique on peut compter les livres numérique
et les documents audiovisuelle électronique.
\item[•] Un utilisateur à un nom, un prénom, nom d'utilisateur, un mot de passe un 
contact, un profil, une adresse et un sexe.
\item[•] Parmi les utilisateurs on peut compter les abonnés qui ont en plus : une date
de naissance, un niveau d'étude, une carte d'identité ou scolaire, une profession et le
contact d'un proche et des personnels (employer de la médiathèque) qui ont un statu en 
plus. 
\item[•] Pour être abonné il faut payer un abonnement selon son niveau par divers moyen
a savoir : mobile money (T-money, Flooz) et le payement en liquide.
\item[•] Un abonné peut effectuer un emprunt ou une réservation et doit restituer
chaque emprunt.
\item[•] Un abonné a la possibilité de télécharger un ouvrage.
\end{itemize}
\newpage
\includegraphics[scale=0.25]{images/CD.png}
\newpage
\subsection{Technologies et outils utilisé}
Au cours notre stage nous avons utiliser divers technologie et outils. Ici nous allons 
entrer plus en détails sur ces technologies et plus important encore nous allons 
expliquer la pertinence et le pourquoi de ces technologies.\\
\subsubsection{Technologie utilisé}
Parmi les technologies utilisé, nous pouvons citer : Laravel , PosttgreSQL et Git.
\begin{itemize}
\item[•] UML : \\
C'est un Langage de Modélisation Unifier. C'est la toute première technologie que nous avons
utilisé il Nous à permis de mieux analyser le problème modéliser le problème au quelle
nous étions confronté. Nous l'avons choisit du fait de sont efficacité dans la modélisation
de données.
\item[•] PostgreSQL : \\
C'est un SGBD (Système de Gestion de base de données) nous avons choisit de l'utiliser
pourpouvoir stocker les données dans une base de donné. Nous l'avons choisi car en
plus d'être une technologie apprise à l'institue, il est robuste, très efficace et plus
sûr.
\item[•] Laravel : \\
C'est un framework du langage de programmation php nous l'avons choisit car c'est une
technologie web et seul les technologie web sont capable de répondre efficacement aux besoins du
client (La médiathèque). De plus Laravel est la seul technologie web qu'on connais.
\end{itemize}
\subsubsection{Outils utilisé}
Parmi les outils utilisé, nous pouvons citer : Git , Planner, PHPStorm, VScode, Machine serveur.
\begin{itemize}
\item[•] Git : \\
C'est un gestionnaire de version. Il nous à permis de travailler efficacement en équipe en nous
permettant notamment de fusionner facilement notre travail et de travailler à distance. 
\item[•] Planner : \\
C'est un outils de gestion de projet. Il nous à permis comme vous pouvez aisément le devine de 
gérer notre projet. Nous grâce à ce dernier pu faire une étude prévisionnelle du projet et faire
la répartition des tâches.
\item[•] PHPStorm \& Vscode : \\
Ce sont des \textbf{IDE (Environnement de Développement Intégré)} Ils nous ont permis de
travailler facilement, efficacement et rapidement sur notre projet mais nous vous conseillons 
d'utiliser PHPStorm car il est fait pour le développent en \textbf{PHP}.
\item[•] Machine Serveur : \\
C'est une machine qui sur la quelle nous avons hébergé notre site web.
\end{itemize}
\section{Réalisation}
\section{Difficulté}
\subsection{Difficulté rencontré}
Au cours de notre stage de six semaine à la médiathèque, nous avons avons rencontré divers 
difficulté. certain d'ordre technique et d'autre d'ordre sociale.
\subsubsection{Difficulté technique}
Parmi les difficulté technique, nous pouvons citer le fait que Laravel n'était pas totalement
adapté pour résoudre notre problème du fais qu'il ne support pas des modèles basé sur l'héritage
de donnée. De ce fait il nous à été très difficile d'implémenter une partie de notre modèle.
Nous étions pas excellent en Laravel nous avons eu quasiment des problème à chaque
fois que nous devions implémenter quelque chose. En effet nous avons eu du mal à implémenté les 
Relation du modèle sûr tous la relation \textbf{plusieurs à plusieurs}, modules comme la barre 
de debugage, le module d'import etexport excel, le module de QrCode et j'en passe. 
 
\subsection{Amélioration future}
A ce jour le projet n'est pas totalement terminé. Donc il est tout à fais logique de dire que
que nous allons déjà terminer notre application avant de commencer à l'améliorer.
Comme amélioration, nous avons prévus d'optimiser notre code pour que les pages se charge 
plus rapidement. Gérer l'abonnement via flooz et T-money.

\section{Conclusion}

\end{document}



